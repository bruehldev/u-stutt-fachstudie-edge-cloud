\documentclass[runningheads,a4paper]{llncs}
\usepackage[verbose]{placeins}
% für neue deutsche Rechtschreibung
\usepackage[ngerman]{babel}
% für englische Rechtschreibung
%\usepackage[american]{babel}

%Eingabeformat UTF-8
\usepackage[utf8]{inputenc}

\usepackage{graphicx}
\graphicspath{{graphics/}}

%Tweaks by IPVS/AS
\usepackage{lncs_as}

%extended enumerate, such as \begin{compactenum}
\usepackage{paralist}

%put figures inside a text
%\usepackage{picins}
%use
%\piccaptioninside
%\piccaption{...}
%\parpic[r]{\includegraphics ...}
%Text...

%Sorts the citations in the brackets
%\usepackage{cite}

%for easy quotations: \enquote{text}
\usepackage{csquotes}

\usepackage[T1]{fontenc}

%enable margin kerning
\usepackage{microtype}

%better font, similar to the default springer font
\usepackage[%
rm={oldstyle=false,proportional=true},%
sf={oldstyle=false,proportional=true},%
tt={oldstyle=false,proportional=true,variable=true},%
qt=false%
]{cfr-lm}
%
%if more space is needed, exchange cfr-lm by mathptmx
%\usepackage{mathptmx}

%for demonstration purposes only
\usepackage[math]{blindtext}

\usepackage{listings}
\lstloadlanguages{java}
\lstset{language=java,numbers=left,captionpos=b}

\usepackage[hyperref,svgnames]{xcolor}

\usepackage[
%pdfauthor={},
%pdfsubject={},
%pdftitle={},
%pdfkeywords={},
bookmarks=false,
breaklinks=true,
colorlinks=true,
linkcolor=black,
citecolor=black,
filecolor=DarkBlue,
urlcolor=DarkBlue,
pdfstartview=Fit,
pdfpagelayout=SinglePage
]{hyperref}
%enables correct jumping to figures when referencing
\usepackage[all]{hypcap}

\usepackage[capitalise,nameinlink,ngerman]{cleveref}
%Nice formats for \cref - only for English texts
%\crefname{section}{Sect.}{Sect.}
%\Crefname{section}{Section}{Sections}
%\crefname{figure}{Fig.}{Fig.}
%\Crefname{figure}{Figure}{Figures}

\usepackage{xspace}
%\newcommand{\eg}{e.\,g.\xspace}
%\newcommand{\ie}{i.\,e.\xspace}
\newcommand{\eg}{e.\,g.,\ }
\newcommand{\ie}{i.\,e.,\ }

%introduce \powerset - hint by http://matheplanet.com/matheplanet/nuke/html/viewtopic.php?topic=136492&post_id=997377
\DeclareFontFamily{U}{MnSymbolC}{}
\DeclareSymbolFont{MnSyC}{U}{MnSymbolC}{m}{n}
\DeclareFontShape{U}{MnSymbolC}{m}{n}{
    <-6>  MnSymbolC5
   <6-7>  MnSymbolC6
   <7-8>  MnSymbolC7
   <8-9>  MnSymbolC8
   <9-10> MnSymbolC9
  <10-12> MnSymbolC10
  <12->   MnSymbolC12%
}{}
\DeclareMathSymbol{\powerset}{\mathord}{MnSyC}{180}

% correct bad hyphenation here
\hyphenation{op-tical net-works semi-conduc-tor}

\begin{document}

%Works on MiKTeX only
%hint by http://goemonx.blogspot.de/2012/01/pdflatex-ligaturen-und-copynpaste.html
%also http://tex.stackexchange.com/questions/4397/make-ligatures-in-linux-libertine-copyable-and-searchable
%This allows a copy'n'paste of the text from the paper
\input glyphtounicode.tex
\pdfgentounicode=1

\title{Analyse existierender Deployment- und Management-Technologien für die Edge Cloud}
%If Title is too long, use \titlerunning
%\titlerunning{Short Title}

\author{Christoph Braun, Daniel Br"uhl}

\supervisor{M.Sc. Karoline Saatkamp / M.Sc. Michael Wurster}

\seminar{IAAS - Fachstudie}

\semester{SS 2017}

\abgabedatum{Stuttgart, 15.10.2017}

\institute{\email{st103425@stud.uni-stuttgart.de}}

\frontpagede % creates the frontpage (in German)
%\frontpageen % creates the frontpage (in English)

\thispagestyle{empty}
%\cleardoublepage
%\pagebreak
%\setcounter{tocdepth}{2}
%\tableofcontents

\maketitle

\begin{abstract}
Hier wird das Abstract stehen
\end{abstract}

%\keywords{...}


\section{Einleitung}



\subsection{Motivation}


\subsection{Aufbau der Ausarbeitung}

\newpage
\bibliographystyle{splncs03}
\bibliography{paper}
\end{document}